
\section* Sesión 1: Introducción al conjunto de los números enteros

\section*# Objetivo de aprendizaje
Reconocer el conjunto de los números enteros y su representación en la recta numérica, aplicando su uso en contextos cotidianos relacionados con programas de administración, gastronomía, ingeniería y salud.

\section*# Contenido teórico

> El sistema de los números enteros está formado por los números del conjunto:
>
> $$\mathbb{Z} = \{ ..., -3, -2, -1, 0, 1, 2, 3, ... \}$$
>
> Incluye números positivos, negativos y el cero, el cual es neutro. 
> Su representación se realiza sobre una recta numérica en donde cada número tiene una posición única.
>
> El valor absoluto se representa como: 
>
> $$|x| = \begin{cases} 
x & \text{si } x \geq 0 \\
-x & \text{si } x < 0
\end{cases}$$
>
> Fuente: Arcila Vanegas & Gómez Noreña, 2016, p. 1–2.

\section*# Cuestionario diagnóstico (retroalimentación automática)

1. ¿Cuál de los siguientes números pertenece al conjunto de los enteros?
   - a) 3.5
   - b) $-\sqrt{2}$
   - c) $-4$ ✅ *(Retro: Correcto. $-4$ es un número entero negativo)*
   - d) $\pi$

2. El valor absoluto de $-8$ es:
   - a) $-8$
   - b) $0$
   - c) $8$ ✅ *(Retro: Correcto. $|-8| = 8$)*

3. Verdadero o Falso: El número $0$ es un número positivo.
   - ❌ *Falso.* *(Retro: Correcto. El $0$ no es positivo ni negativo.)*

\section*# Tarea Moodle (retroalimentación automática)

1. Ordena de menor a mayor: $-5, 3, 0, -1, 7$
   ✅ $-5, -1, 0, 3, 7$

2. Representa en la recta numérica: $-3, 0, 4, -1$

\section*# Juego Moodle sugerido
Crucigrama con los términos:
- Valor absoluto
- Enteros positivos
- Enteros negativos
- Cero
- Recta numérica

\section*# Cita APA 7
Arcila Vanegas, M. D., & Gómez Noreña, Y. E. (2016). *Aritmética: Teoría, ejemplos y problemas* (1a ed., pp. 1–2). Instituto Tecnológico Metropolitano – ITM, Colegio Mayor de Antioquia.

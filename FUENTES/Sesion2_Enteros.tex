
\section* Sesión 2: Adición y sustracción con números enteros

\section*# Objetivo de aprendizaje
Realizar operaciones de suma y resta con números enteros aplicando correctamente las reglas de los signos, identificando su utilidad en contextos del día a día y de las profesiones.

\section*# Contenido teórico

> La **adición** de dos enteros se calcula según sus signos:
>
> - **Mismo signo**: se suman los valores absolutos y se conserva el signo.
> - **Distinto signo**: se resta el mayor menos el menor en valor absoluto y se conserva el signo del mayor.
>
> La **sustracción** se transforma en una suma del opuesto:  
> $$a - b = a + (-b)$$
>
> Ejemplos:
> - $(-5) + (-3) = -8$
> - $7 + (-4) = 3$
> - $5 - 8 = -3$
> - $-4 - (-6) = -4 + 6 = 2$
>
> Fuente: Arcila Vanegas & Gómez Noreña, 2016, pp. 2–4.

\section*# Aplicaciones por programa

- **Turismo y gastronomía**: Cambios de temperatura o pérdidas de inventario.
- **Ingeniería**: Cálculos de niveles subterráneos o diferencias de potencial.
- **Administración y comercio**: Balance de ingresos y egresos.

\section*# Cuestionario (retroalimentación automática)

1. ¿Cuál es el resultado de $(-5) + 3$?
   - a) 8
   - b) -2 ✅ *(Retro: Correcto. Se resta 5 - 3 = 2 y se toma el signo del mayor en valor absoluto)*
   - c) 2
   - d) -8

2. ¿Cuál es el resultado de $6 - 9$?
   - a) 3
   - b) -3 ✅ *(Retro: Correcto. Se resta 9 - 6 = 3, y el resultado es negativo.)*
   - c) -15
   - d) 15

3. ¿Cuál es el resultado de $-3 - (-2)$?
   - a) -5
   - b) -1 ✅ *(Retro: Correcto. Se convierte en $-3 + 2 = -1$)*
   - c) 5
   - d) 1

\section*# Tarea Moodle (con retroalimentación)

1. Realiza las siguientes operaciones:
   - a) $-7 + 5$ → ✅ Respuesta: $-2$
   - b) $6 - 9$ → ✅ Respuesta: $-3$
   - c) $-3 - (-2)$ → ✅ Respuesta: $-1$

\section*# Juego Moodle sugerido
"Ahorcado" con términos:
- Adición
- Sustracción
- Valor absoluto
- Signo
- Negativo

\section*# Cita APA 7
Arcila Vanegas, M. D., & Gómez Noreña, Y. E. (2016). *Aritmética: Teoría, ejemplos y problemas* (1a ed., pp. 2–4). Instituto Tecnológico Metropolitano – ITM, Colegio Mayor de Antioquia.

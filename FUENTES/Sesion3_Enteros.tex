
\section* Sesión 3: Multiplicación y división de números enteros

\section*# Objetivo de aprendizaje
Aplicar correctamente las reglas de los signos en las operaciones de multiplicación y división con números enteros, relacionando su uso en contextos académicos y laborales.

\section*# Contenido teórico

> Las operaciones de **multiplicación** y **división** entre enteros siguen las siguientes reglas de signos:
>
> - Signos **iguales**: resultado **positivo**
> - Signos **diferentes**: resultado **negativo**
>
> Ejemplos:
> - $(-4) \times (-2) = 8$
> - $(-9) \div 3 = -3$
> - $(+6) \div (-2) = -3$
> - $(-12) \times (+1) = -12$
>
> Fuente: Arcila Vanegas & Gómez Noreña, 2016, p. 4.

\section*# Aplicaciones por programa

- **Gastronomía y procesos**: Cálculo de variaciones de temperatura por tanda de hornos.
- **Turismo y guianza**: Costos por grupo, ingresos por unidad.
- **Ingeniería y arquitectura**: Cálculo de cargas o resistencias con signos.
- **Salud**: Cambios de dosis o promedios con valores negativos.

\section*# Cuestionario (retroalimentación automática)

1. ¿Cuál es el resultado de $-6 \times 4$?
   - a) 24
   - b) -24 ✅ *(Retro: Correcto. Signos diferentes dan resultado negativo.)*
   - c) -10
   - d) 10

2. ¿Cuál es el resultado de $-15 \div (-5)$?
   - a) -3
   - b) 3 ✅ *(Retro: Correcto. Signos iguales dan resultado positivo.)*
   - c) -10
   - d) 0

3. ¿Cuál es el resultado de $12 \div (-3)$?
   - a) 4
   - b) -4 ✅ *(Retro: Correcto. Signos diferentes, resultado negativo.)*
   - c) 0
   - d) -3

\section*# Tarea Moodle (con retroalimentación)

1. Realiza las siguientes operaciones:
   - a) $-6 \times 4$ → ✅ Respuesta: $-24$
   - b) $-15 \div (-5)$ → ✅ Respuesta: $3$
   - c) $12 \div (-3)$ → ✅ Respuesta: $-4$

\section*# Juego Moodle sugerido
"Millonario" con preguntas de verdadero/falso y selección múltiple sobre signos.

\section*# Cita APA 7
Arcila Vanegas, M. D., & Gómez Noreña, Y. E. (2016). *Aritmética: Teoría, ejemplos y problemas* (1a ed., p. 4). Instituto Tecnológico Metropolitano – ITM, Colegio Mayor de Antioquia.


\section* Sesión 4: Propiedades de la suma y del producto

\section*# Objetivo de aprendizaje
Reconocer y aplicar las propiedades conmutativa, asociativa, distributiva y factor común en operaciones con números enteros.

\section*# Contenido teórico

> Las propiedades fundamentales de la suma y la multiplicación permiten simplificar y reorganizar operaciones:
>
> - **Conmutativa**:
>   - Suma: $a + b = b + a$
>   - Producto: $a \times b = b \times a$
>
> - **Asociativa**:
>   - Suma: $(a + b) + c = a + (b + c)$
>   - Producto: $(a \times b) \times c = a \times (b \times c)$
>
> - **Distributiva**:
>   - $a \times (b + c) = a \times b + a \times c$
>
> - **Factor común**:
>   - $ab + ac = a(b + c)$
>
> Fuente: Arcila Vanegas & Gómez Noreña, 2016, pp. 5–6.

\section*# Aplicaciones por programa

- **Administración y comercio**: Agrupar productos por unidades comunes.
- **Gastronomía**: Agrupación de ingredientes que se repiten en recetas.
- **Ingeniería**: Simplificación de expresiones de cálculo de fuerzas o resistencias.

\section*# Cuestionario (retroalimentación automática)

1. ¿Cuál propiedad se aplica en la expresión $(2 + 3) + 4 = 2 + (3 + 4)$?
   - a) Conmutativa
   - b) Asociativa ✅ *(Retro: Correcto. Agrupación de sumandos sin alterar el resultado.)*
   - c) Distributiva
   - d) Factor común

2. ¿Cuál es el resultado de aplicar la distributiva a $5 \times (4 + 3)$?
   - a) $20 + 3$
   - b) $5 + (4 + 3)$
   - c) $5 \times 4 + 5 \times 3 = 20 + 15 = 35$ ✅ *(Retro: Correcto.)*
   - d) $7 \times 5$

3. ¿Cómo se factoriza la expresión $3x + 3y$?
   - a) $3(x + y)$ ✅ *(Retro: Correcto. Se aplica factor común.)*
   - b) $x(3 + y)$
   - c) $3xy$
   - d) $x + 3y$

\section*# Tarea Moodle (con retroalimentación)

1. Usa la propiedad distributiva para simplificar: $5 \times (4 + 3)$ → ✅ Respuesta: $35$
2. Factoriza: $3x + 3y$ → ✅ Respuesta: $3(x + y)$

\section*# Juego Moodle sugerido
"Crucigrama" con los términos:
- Conmutativa
- Asociativa
- Distributiva
- Factor común

\section*# Cita APA 7
Arcila Vanegas, M. D., & Gómez Noreña, Y. E. (2016). *Aritmética: Teoría, ejemplos y problemas* (1a ed., pp. 5–6). Instituto Tecnológico Metropolitano – ITM, Colegio Mayor de Antioquia.

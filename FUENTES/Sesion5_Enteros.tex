
\section* Sesión 5: Signos de agrupación y jerarquía de operaciones

\section*# Objetivo de aprendizaje
Resolver expresiones numéricas utilizando correctamente los signos de agrupación y aplicando la jerarquía de operaciones.

\section*# Contenido teórico

> En matemáticas, el orden correcto para resolver operaciones se conoce como **jerarquía de operaciones**. El uso de signos de agrupación ($(), [], \{\}$) ayuda a organizar y priorizar cálculos.
>
> **Jerarquía**:
> 1. Paréntesis y agrupaciones
> 2. Potencias y raíces
> 3. Multiplicación y división
> 4. Suma y resta
>
> Ejemplo:
> $$3 + 2 \times (5 - 3) = 3 + 2 \times 2 = 3 + 4 = 7$$
>
> $$ (4 + 3) \times (2 - 1) = 7 \times 1 = 7 $$
>
> Fuente: Arcila Vanegas & Gómez Noreña, 2016, p. 5.

\section*# Aplicaciones por programa

- **Gastronomía**: Costeo de recetas con múltiples ingredientes y pasos.
- **Turismo**: Presupuestos con descuentos, impuestos y tarifas escalonadas.
- **Ingeniería**: Cálculos de estructuras con expresiones compuestas.
- **Salud**: Cálculo de dosis o indicadores con varias operaciones secuenciales.

\section*# Cuestionario (retroalimentación automática)

1. ¿Cuál operación se debe resolver primero en $5 + [3 \times (2 + 1)]$?
   - a) $5 + 3$
   - b) $2 + 1$ ✅ *(Retro: Correcto. Los paréntesis tienen prioridad.)*
   - c) $3 \times 2$
   - d) $5 + [3 \times 3]$

2. ¿Resultado final de $(6 + 2) \times (4 - 1)$?
   - a) 6
   - b) 8
   - c) 24 ✅ *(Retro: Correcto. Primero se resuelven los paréntesis: $8 \times 3 = 24$)*
   - d) 3

3. ¿Qué indica el uso de corchetes y llaves?
   - a) Separar resultados
   - b) Dar estilo
   - c) Jerarquizar agrupaciones ✅ *(Retro: Correcto. Agrupan operaciones para jerarquizar cálculos.)*
   - d) Eliminar operaciones

\section*# Tarea Moodle (con retroalimentación)

1. Evalúa: $5 + [3 \times (2 + 1)]$ → ✅ Respuesta: $14$
2. Resuelve: $(6 + 2) \times (4 - 1)$ → ✅ Respuesta: $24$

\section*# Juego Moodle sugerido
"Millonario" o "Crucigrama" con jerarquía de operaciones y agrupaciones.

\section*# Cita APA 7
Arcila Vanegas, M. D., & Gómez Noreña, Y. E. (2016). *Aritmética: Teoría, ejemplos y problemas* (1a ed., p. 5). Instituto Tecnológico Metropolitano – ITM, Colegio Mayor de Antioquia.


\section* Sesión 6: Polinomios aritméticos

\section*# Objetivo de aprendizaje
Resolver expresiones numéricas compuestas (polinomios aritméticos) con números enteros, aplicando operaciones combinadas y jerarquía.

\section*# Contenido teórico

> Un **polinomio aritmético** es una expresión numérica que combina múltiples operaciones: suma, resta, multiplicación y división.
> 
> Para resolverlos, se deben seguir las reglas de la jerarquía de operaciones y respetar los signos de agrupación.
>
> Ejemplo:
> $$2 + 3 \times (-4) - [6 - (-2)] = 2 - 12 - [6 + 2] = -10 - 8 = -18$$
>
> Fuente: Arcila Vanegas & Gómez Noreña, 2016, p. 6.

\section*# Aplicaciones por programa

- **Turismo y gastronomía**: Combinación de costos, descuentos e impuestos.
- **Ingeniería**: Cálculos técnicos con varias operaciones interrelacionadas.
- **Salud**: Cálculo de indicadores que requieren múltiples pasos aritméticos.

\section*# Cuestionario (retroalimentación automática)

1. ¿Qué se debe hacer primero en $2 + 3 \times (-4) - [6 - (-2)]$?
   - a) Sumar 2 y 3
   - b) Multiplicar $3 \times (-4)$ ✅ *(Retro: Correcto. Primero se realizan multiplicaciones.)*
   - c) Sumar 6 y 2
   - d) Restar 2 de 6

2. ¿Resultado de $5 + 3 \times (-2) - [4 + (-6)]$?
   - a) -5
   - b) -7 ✅ *(Retro: Correcto. $5 - 6 - [4 - 6] = -1 - (-2) = -1 + 2 = 1$)*
   - c) 1
   - d) 7

3. ¿Resultado de $(-3)^2 + 2 \times (-4) + 7$?
   - a) 9 - 8 + 7 = 8 ✅ *(Retro: Correcto. Exponente antes que multiplicación.)*
   - b) -9 - 8 + 7
   - c) -17 + 7
   - d) 0

\section*# Tarea Moodle (con retroalimentación)

1. Resuelve: $5 + 3 \times (-2) - [4 + (-6)]$ → ✅ Respuesta: $-7$
2. Evalúa: $(-3)^2 + 2 \times (-4) + 7$ → ✅ Respuesta: $8$

\section*# Juego Moodle sugerido
"Crucigrama" con términos como: polinomio, signo, agrupación, jerarquía, entero.

\section*# Cita APA 7
Arcila Vanegas, M. D., & Gómez Noreña, Y. E. (2016). *Aritmética: Teoría, ejemplos y problemas* (1a ed., p. 6). Instituto Tecnológico Metropolitano – ITM, Colegio Mayor de Antioquia.
